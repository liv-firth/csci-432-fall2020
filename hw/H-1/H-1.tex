\documentclass{article}
\usepackage{../fasy-hw}
\usepackage{ wasysym }

%% UPDATE these variables:
\renewcommand{\hwnum}{1}
\title{Advanced Algorithms, Homework 1}
\author{Olivia Firth}
\collab{n/a}
\date{due: 27 August 2020}

\begin{document}

\maketitle

This homework assignment is due on 27 August 2020, and should be
submitted as a single PDF file to D2L and to Gradescope.

General homework expectations:
\begin{itemize}
    \item Homework should be typeset using LaTex.
    \item Answers should be in complete sentences and proofread.
\end{itemize}

\nextprob
\collab{n/a}

Answer the following questions:
\begin{enumerate}
    \item What is your elevator pitch?  Describe yourself in 1-2
                sentences.
                \newline \textbf{I am a computer science graduate student with a background in mathematics, focusing on education. In my free time, I enjoy 
                climbing, biking, skiing, and making art.}
    \item What was your favorite CS class so far, and why?
    \newline \textbf{Computer Science Theory. This was my favorite course because it provided a fun connection 
    between my math background and computer science. I also really enjoy the way Sean Yaw teaches. }
    \item What was your least favorite CS class so far, and why?
    \newline \textbf{My least favorite course was Computer Systems. I don't think this was a fault of the course, but I struggled 
    to get as much out of it as I probably could because I was simultaneously trying to teach myself C++.}
    \item Why are you interested in taking this course?
    \newline \textbf{It seemed like an interesting exploration of the mathematically thinking behind computer science.}
    \item What is your biggest academic or research goal for this semester (can
        be related to this course or not)?
        \newline \textbf{My biggest goal this semester is to determine a more clear direction for my graduate studies.}
    \item What do you want to do after you graduate?
    \newline \textbf{I am interested in getting some teaching experience and potentially working on my Ph.D.}
    \item What was the most challenging aspect of your coursework last semester
        after the university transitioned to online?
        \newline \textbf{I struggled to create structure for myself and also had a hard time learning without the ability to collaborate with other students or the professor in person. }
    \item What went well last semester for you after the university transitioned
        to online?
        \newline \textbf{It is easier for me to focus on lectures when they are virtual and especially when they are recorded and I can go back to things that I may have missed. }
\end{enumerate}

\paragraph{Answer}

% ============================================

% ============================================

\nextprob
\collab{n/a}

Please do the following:
\begin{enumerate}
    \item Write this homework in LaTex.
        Note: if you have not used LaTex before and this is an
        issue for you, please contact the instructor or TA.
    \item Update your photo on D2L to be a recognizable headshot of you.
    \item Sign up for the class discussion board.
\end{enumerate}

\paragraph{Answer}

% ============================================

I have completed these tasks. 

% ============================================


\nextprob
\collab{}

    In this class,
    please properly cite all resources that you use.
    To refresh your memory on what plagiarism is,
    please
    complete the plagiarism tutorial found here:
    \url{http://www.lib.usm.edu/plagiarism_tutorial}.
    If you have observed plagiarism or cheating in a classroom (either as an
    instructor or as a student), explain the situation and how it made you
    feel.  If you have not experienced plagiarism or cheating or if you would
    prefer not to reflect on a personal experience, find a news
    article about plagiarism or cheating and explain how you would feel if you
    were one of the people involved.


\paragraph{Answer}

% ============================================

This is an instance of cheating more that plagiarism, but in eighth grade my math teacher stopped an exam because the kid sitting next to me 
had apparently been looking at my test. I did not notice that it was happening, but my teacher did. It made me really uncomfortable and now I get a 
little paranoid when I take tests in close quarters. I am glad that my teacher didn't accuse me of participating in the cheating by allowing my table-mate to copy. 
I just remember being very surprised and confused. 


% ============================================



\nextprob
Prove the following statement: Every tree with one or more nodes/vertices has
exactly $n-1$ edges.

\paragraph{Answer}

% ============================================

We are to show that every tree with one or more nodes/vertices has exaclty $n-1$ edges. To do so we will use a 
proof by induction. 

Base case: We show that a tree $n = 1$ nodes, has $n - 1$ edges. For $n = 1$, this means our tree should have 
zero edges. This must be true because an edge is defined as connecting two vertices (Nykamp). 

Inductive Assumption: We assume that the statement is true for any tree with $n = k$ nodes.

Inductive Step: We know that some tree with $k$ nodes has $k - 1$ edges. If we add a node to our tree,
which now has $k + 1$ nodes, we must also add and edge because trees are by definition connected (Weisstein), bring our edges to a total of $k$. 
So, our new tree has $k + 1$ nodes and $k$ edges. 

Conclusion: $k + 1 - 1 = k$, therefore, our tree with $n = k + 1$ nodes has $n -1$ edges, and we have proven our statement to be true. 


References: 
\newline Nykamp DQ, “Edge definition.” From Math Insight. \url{http://mathinsight.org/definition/network_edge}

 Weisstein, Eric W. "Tree." From MathWorld--A Wolfram Web Resource. \url{https://mathworld.wolfram.com/Tree.html}


% ============================================



\nextprob
Use the definition of big-O notation to prove that $f(x)=n^2 + 3n +2$ is
$O(n^2)$.

\paragraph{Answer}

% ============================================
We are to show that $f(x)=n^2 + 3n +2$ is
$O(n^2)$.

We have a function $f(x)=n^2 + 3n +2$ and some function $g(x) = n^2$, both defined on the same set of real numbers. 

By the definition of big-O notation, $f(x) = O(g(x))$ as x goes to infinity, if and only if there exist constansts $N$ and 
$C$ such that $|f(x)| \leq C|g(x)|$ for all $x > N$ (MIT). In other words, $f(x) = O(g(x))$ is only true if $f(x)$ never groes faster than $g(x)$. 

$<===$
\newline Let $N = 1$. Consider the case in which, $|f(x)| =  C|g(x)|$. This means for some $C$, $|n^2 + 3n +2| = C|n^2|$ and therefore, $ C = \frac{n^2 + 3n + 2}{n^2}$. Consider $n = 1$, in this case, $C = 6$.  When we consider any $x > N$ where we've declared $N = 1$, $\frac{n^2 + 3n + 2}{n^2}$ will always be less than 6. 
As $n$ grows to infinity, the $n^2$ term grows progressively faster than $3n$, thus,  $C = 6$ satisfies our equation such that $|f(x)| \leq C|g(x)|$ for all $x > N$. 




References: 

\url{https://web.mit.edu/16.070/www/lecture/big_o.pdf} - I couldn't find an author for this source, but it seemed reliable. I refer to it as MIT in my solution. 

% ============================================



\nextprob
Consider the \textsc{RightAngle} algorithm on page 8 of the textbook.
\begin{enumerate}
    \item When we design an algorithm, we design the algorithm to solve a
        problem or answer a question.  What is the problem that this algorithm
        solves?
    \item Prove that the while loop terminates.
\end{enumerate}

\paragraph{Answer}

% ============================================

\begin{enumerate}
    \item The \textsc{RightAngle} solves the problem of how to construct a line perpendicular to an input line $l$ and passing through input point $P$, with just a compass and straight edge. 
    \item We are to show that the while loop in \textsc{RightAngle} terminates. To prove that a while loop terminates we need to show that either the condition of the
    the while loop is at some point false, that there is some kind of loop variant, or that it happens in constant time. Each step takes in a finite set of parameters to perform a single function. Therefore, because each 
    step happens in constant time, the algorithm happens in constant time. Thus, the while loop must terminate at some point.  

\end{enumerate}

% ============================================



\nextprob
Consider the following statement: If $a$ and $b$ are both even numbers, then $ab$ is
an even number.
\begin{enumerate}
    \item What is the definition of an odd number?
    \item What is the definition of an even number?
    \item What is the contrapositive of this statement?
    \item What is the converse of this statement?
    \item Prove this statement.
\end{enumerate}

\paragraph{Answer}

% ============================================
\begin{enumerate}
    \item An odd number is an integer of the the form n = 2k + 1, where k is an integer. (Weisstein)
    \item An even number is an integer of the form n = 2k, where k is an interger. (Weisstein)
    \item If $ab$ is not an even number, then neither $a$ or $b$ are even numbers. 
    \item If $ab$ is an even number, then $a$ and $b$ are both even numbers. 
    \item We are to show that If $a$ and $b$ are both even numbers, then $ab$ is
    an even number.

    Suppose we have even numbers $a$ and $b$ such that $a = 2k$ and $b = 2j$ where $k$ and $j$ are both integers (satifsying the defiition of an even number).

    Consider $a*b = 2*k*2*j$. Multiplication is commutative so we can make this $a*b = 2(2*k*j)$. We know that an integer multiplied by an integer is an integer, so let $x = 2*j*k$, where $x$ is an integer. 

    We now have $a*b = 2*x$. This satifies the definition of an even number, therefore we have shown that if $a$ and $b$ are both even numbers, then $ab$ is
    an even number.
\end{enumerate}
References: 

Weisstein, Eric W. "Even Number." From MathWorld--A Wolfram Web Resource. \url{https://mathworld.wolfram.com/EvenNumber.html}

Weisstein, Eric W. "Odd Number." From MathWorld--A Wolfram Web Resource. \url{https://mathworld.wolfram.com/OddNumber.html}
% ============================================



\end{document}

